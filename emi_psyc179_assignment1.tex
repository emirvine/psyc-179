\documentclass{article}
\usepackage{enumitem}
\usepackage{graphicx}
\usepackage{amssymb}
\usepackage{amsmath}
\usepackage[letterpaper, portrait, margin=1in]{geometry}

\SetLabelAlign{parright}{\parbox[t]{\labelwidth}{\raggedleft#1}}
\setlist[description]{style=multiline,topsep=5pt,leftmargin=1.5cm,align=parleft}
\setlength\parindent{0pt}
 
\begin{document}
\subsection*{Emily Irvine \newline PSYC 179 \newline Winter 2016}
\subsection*{
\begin{center}
Assignment 1
\end{center}
}

\setlength{\parindent}{8ex}
For the purpose of this first assignment, my intent is to load my own neural data and perform some basic analysis resulting in tuning curves of neural firing rate based on position location. Specifically, I aim to do this in Python, an open-source programming language.

The data that I am using in this assignment was from an experiment I conducted in 2015. Briefly, we are interested in whether dorsal hippocampus CA1 neurons represent a behaviorally-relevant space that the animal has not previously experienced. Specifically, rats were exposed to three main experimental phases. In Phase 1 the animals run between two feeders dispensing two food pellets along a "`U"' trajectory. In Phase 2 a shortcut and novel non-shortcut trajectory is introduced but is not yet accessible to the animals. Only during Phase 3 are the animals able to explore the shortcut and novel trajectories. 

There are currently no loading functions to import Neuralynx data directly to Python. Hopefully that will change, but for now I import the data through Matlab's loading functions to load them into Python. I show that I can load my data through plots. Next, I show that I am able to slice the data in time, which is a common operation for our data analysis.

\end{document}